

%2multibyte Version: 5.50.0.2953 CodePage: 1252
%\newtheorem{remark}[theorem]{Remark}
% \gamemathtrue
%\usepackage{array}
%\usepackage{tabularx}
%\usepackage[active]{srcltx}
%\usepackage[active]{srcltx}
%\usepackage{ae}
%\usepackage{aecompl}
%\usepackage{amssymb}
%\usepackage{xcolor}
%\usepackage{amsmath}
%\usepackage{amsfonts}
%\usepackage{graphicx}
%\usepackage{accents}
%\usepackage{natbib}
%\usepackage[colorlinks,linkcolor=links,citecolor=cites,urlcolor=MyDarkBlue]{hyperref}
%\usepackage{array}
%\usepackage{sgame}
%\usepackage{showkeys}
%\theoremstyle{plain}
%\usepackage{showkeys}
% \usepackage[pagebackref,colorlinks,urlcolor=black,citecolor=black,filecolor=black,
% linkcolor=black, bookmarksnumbered,bookmarksopen=true,bookmarksopenlevel=0]%
% {hyperref}%

\documentclass[12pt,notitlepage]{article}%
\usepackage[top=1.5in, bottom=1.5in, left=1.5in, right=1.5in]{geometry}
\usepackage{amsthm}
\usepackage{setspace}
\usepackage{eurosym}
\usepackage{subfigure}
\usepackage{xcolor}
\usepackage{url}
\usepackage{supertabular}
\usepackage{amssymb}
\usepackage{graphicx}
\usepackage[colorlinks,linkcolor=links,citecolor=cites,urlcolor=MyDarkBlue]%
{hyperref}
\usepackage{amsfonts}
\usepackage{amsmath}
\usepackage{amstext}
\usepackage{appendix}
%\usepackage{mathpazo}
\usepackage{tikz}
\usepackage{rotating}
\usepackage{lscape}
\usepackage{chngpage}
\usepackage{esint}
\usepackage{natbib}
\usepackage{bm}
\setcounter{MaxMatrixCols}{30}
%TCIDATA{OutputFilter=latex2.dll}
%TCIDATA{Version=5.50.0.2953}
%TCIDATA{Codepage=1252}
%TCIDATA{CSTFile=article.cst}
%TCIDATA{Created=Monday, November 05, 2001 21:57:33}
%TCIDATA{LastRevised=Saturday, October 12, 2013 14:43:03}
%TCIDATA{<META NAME="GraphicsSave" CONTENT="32">}
%TCIDATA{<META NAME="SaveForMode" CONTENT="1">}
%TCIDATA{BibliographyScheme=Manual}
%TCIDATA{<META NAME="DocumentShell" CONTENT="Articles\SW\AMS Journal Article">}
%TCIDATA{Language=American English}
%BeginMSIPreambleData
\providecommand{\U}[1]{\protect\rule{.1in}{.1in}}
%EndMSIPreambleData
\newtheorem{theorem}{Theorem}
\newtheorem{acknowledgement}{Acknowledgement}
\newtheorem{algorithm}{Algorithm}
\newtheorem{Assumption}{Assumption}
\newtheorem{axiom}{Axiom}
\newtheorem{case}{Case}
\newtheorem{claim}{Claim}
\newtheorem{conclusion}{Conclusion}
\newtheorem{condition}{Condition}
\newtheorem{conjecture}{Conjecture}
\newtheorem{corollary}{Corollary}
\newtheorem{criterion}{Criterion}
\newtheorem{definition}{Definition}
\newtheorem{example}{Example}
\newtheorem{exercise}{Exercise}
\newtheorem{lemma}{Lemma}
\newtheorem{notation}{Notation}
\newtheorem{problem}{Problem}
\newtheorem*{assumption}{Asumption}
\newtheorem{proposition}{Proposition}
\newtheorem{remark}{Remark}
\newtheorem{solution}{Solution}
\newtheorem{summary}{Summary}
\newtheorem{observation}{Observation}
\newcommand{\RNum}[1]{\uppercase\expandafter{\romannumeral #1\relax}}
\numberwithin{equation}{section}
\DeclareMathOperator*{\argmax}{argmax}
\DeclareMathOperator*{\Prob}{Prob}
\setlength{\topmargin}{0in}
\setlength{\textheight}{8.8in}
\setlength{\oddsidemargin}{0.1in}
\setlength{\evensidemargin}{0.1in}
\setlength{\textwidth}{6.5in}
\setlength{\headheight}{0in}
\def \definitionname{Definition}
\def \sectionautorefname{Section}
\def \subsectionautorefname{Section}
\def \footnotename{footnote}
\def \examplename{Example}
\def \lemmaname{Lemma}
\def \propositionname{Proposition}
\def \appendixname{Appendix}
\def \assumptionname{Assumption}
\def \corollaryname{Corollary}
\def \remarkname{Remark}
\def \remname{Remark}
\providecommand{\possessivecite}[1]{\citeauthor{#1}'s\nolinebreak[2]
(\citeyear{#1})}
\definecolor{MyDarkBlue}{rgb}{0,0.08,0.45}
\definecolor{cites}{HTML}{324b13}
\definecolor{links}{HTML}{1a663b}
\definecolor{MyLightMbuyera}{cmyk}{0.1,0.8,0,0.1}
\hypersetup{
colorlinks,citecolor=blue,filecolor=black,linkcolor=blue,urlcolor=blue
}
%\doublespace
%\baselineskip 2.2em
\parindent= 0.6cm
%%\setlength{\parskip}{0.8ex}
\linespread{1.3}

\usepackage{chngcntr}
\counterwithout{equation}{section}
\begin{document}
\title{Problem set 8%
		}


\maketitle


\section{Problem 1}

Edgeworth box. Consider a pure exchange economy with two kinds of goods and two consumers. Each consumer is endowed with $\bf{one}$ unit of good 1 and $\bf{1/2}\;(a \;half)$ unit of good 2. Their consumption sets are each $\mathbb R_{+}^2$, and their utility functions $u_1$ and $u_2$ are defined by, for each consumption bundle $(x,y)\in R_{+}^2$, 
\begin{equation}
\begin{split}
u_1(x,y):=&x^2+y^2,\\
u_2(x,y):=&x+y,
\end{split}
\end{equation}

	
	
	a. For each of the following items, write down the explicit solution for this economy, in
set-theoretic notations, and label the solution in a diagram with clearly marked axes,
origins, coordinates and indifference maps:
\begin{itemize}
	\item[i.] The set of all Pareto optimal allocations
	\item[ii.] The set of all Walras equilibria allocations (be sure to specify the supporting price vectors)
price vectors)
\item[iii.] The core
	\end{itemize}


b. For each of the following theorems, point out whether the theorem is applicable to this
economy and explain your answer concisely, in no more than 20 words (i.e., for each
theorem, check whether all its conditions are satisfied or not and briefly explain why)
\begin{itemize}
	\item[i.] The first fundamental theorem of welfare economics
	\item[ii.] The second fundamental theorem of welfare economics
	\end{itemize}
	
	
	
	
	
	
	

\section{Problem 2}
Edgeworth box. Consider a pure exchange economy with two kinds of goods and two consumers. Each consumer is endowed with $\bf two$ unit of good 1 and $\bf one$ unit of good 2. Their consumption sets are each $\mathbb R_{+}^2$, and their utility functions $u_1$ and $u_2$ are defined by, for each consumption bundle $(x,y)\in R_{+}^2$,
\begin{equation}
\begin{split}
u_1(x,y):=&\lfloor  x+y\rfloor,\\
u_2(x,y):=&x+2y,
\end{split}
\end{equation}
where $\lfloor  x+y\rfloor$ stands for the integer part of $x+y$. (E.g., $\lfloor  0.2+0.3\rfloor=\lfloor  0.5\rfloor=0$, $\lfloor  0.7+0.8\rfloor=\lfloor  1.5\rfloor=1$, $\lfloor  1.1+1.9\rfloor=\lfloor  3.0\rfloor=3$. ) 

	a. For each of the following items, write down the explicit solution for this economy, in
set-theoretic notations, and label the solution in a diagram with clearly marked axes,
origins, coordinates and indifference maps:
\begin{itemize}
	\item[i.] The set of all Pareto optimal allocations
	\item[ii.] The set of all Walras equilibrium allocations (be sure to specify the supporting price vectors)
	\item[iii.] The set of all price equilibria with transfers (be sure to specify the supporting
	price vectors)
	\item[iv.] The core
\end{itemize}


b. For each of the following theorems, point out whether the theorem is applicable to this
economy and explain your answer concisely, in no more than 20 words (i.e., for each
theorem, check whether all its conditions are satisfied or not and briefly explain why)
\begin{itemize}
	\item[i.] The first fundamental theorem of welfare economics
	\item[ii.] The second fundamental theorem of welfare economics
\end{itemize}


\section{Problem 3}
Edgeworth box. Consider a pure exchange economy with two kinds of goods and two consumers. Consumer 1 is endowed with $\bf{one}$ unit of good 1 and $\bf{zero}$ unit of good 2; Consumer 2 is endowed with $\bf{one}$ unit of good 1 and $\bf{one}$ unit of good 2. Their consumption sets are each $\mathbb R_{+}^2$, and their utility functions $u_1$ and $u_2$ are defined by, for each consumption bundle $(x,y)\in R_{+}^2$,
\begin{equation}
\begin{split}
u_1(x,y):=&\max\{x,y\},\\
u_2(x,y):=&\min\{x,y\}.
\end{split}
\end{equation}
 
 a. On a clearly labeled Edgeworth box, first locate the coordinate of the endowment point
and label it by E, then graph the indifference curves corresponding to the following
equations (and be precise about the coordinate position of each graph).
 \begin{equation}
 \begin{split}
 u_1(x_1,y_1):=&1/4 \;\;for\;\; consumer \;\;1,\\
 u_2(x_2,y_2):=&1/4 \;\;for\;\; consumer \;\;2.
 \end{split}
 \end{equation}
 
 

b. For each of the following items, write down the explicit solution for this economy, in
set-theoretic notations, and label the solution in a diagram with clearly marked axes,
origins, coordinates and indifference maps:
\begin{itemize}
	\item[i.] The set of all Pareto optimal allocations
	\item[ii.] The set of all Walras equilibrium allocations (be sure to specify the supporting price vectors)
	\item[iii.] The set of all price equilibria with transfers (be sure to specify the supporting
	price vectors)
	\item[iv.] The core
\end{itemize}


c. For each of the following theorems, point out whether the theorem is applicable to this
economy and explain your answer concisely, in no more than 20 words (i.e., for each
theorem, check whether all its conditions are satisfied or not and briefly explain why)
\begin{itemize}
	\item[i.] The first fundamental theorem of welfare economics
	\item[ii.] The second fundamental theorem of welfare economics
\end{itemize}

\end{document}