%% LyX 2.3.6.2 created this file.  For more info, see http://www.lyx.org/.
%% Do not edit unless you really know what you are doing.
\documentclass[10pt,english]{article}
\renewcommand{\rmdefault}{cmr}
\renewcommand{\sfdefault}{cmss}
\renewcommand{\ttdefault}{cmtt}
\renewcommand{\familydefault}{\rmdefault}
\usepackage[latin9]{inputenc}
\usepackage[a4paper]{geometry}
\geometry{verbose,tmargin=1.25in,bmargin=1.25in,lmargin=1.25in,rmargin=1.25in}
\usepackage{babel}
\usepackage{amsmath}
\usepackage{amsthm}
\usepackage{amssymb}
\usepackage{esint}
\usepackage[authoryear]{natbib}
\usepackage[unicode=true,pdfusetitle,
 bookmarks=true,bookmarksnumbered=false,bookmarksopen=false,
 breaklinks=false,pdfborder={0 0 0},pdfborderstyle={},backref=false,colorlinks=false]
 {hyperref}

\makeatletter
%%%%%%%%%%%%%%%%%%%%%%%%%%%%%% Textclass specific LaTeX commands.
\theoremstyle{definition}
\newtheorem{problem}{\protect\problemname}

%%%%%%%%%%%%%%%%%%%%%%%%%%%%%% User specified LaTeX commands.
\usepackage{babel}
\usepackage{babel}
\usepackage{babel}





  \providecommand{\assumptionname}{Assumption}
  \providecommand{\axiomname}{Axiom}
  \providecommand{\claimname}{Claim}
  \providecommand{\definitionname}{Definition}
  \providecommand{\lemmaname}{Lemma}
  \providecommand{\propositionname}{Proposition}
  \providecommand{\remarkname}{Remark}
\providecommand{\corollaryname}{Corollary}
\providecommand{\theoremname}{Theorem}



  \providecommand{\axiomname}{Axiom}
  \providecommand{\claimname}{Claim}
  \providecommand{\definitionname}{Definition}
  \providecommand{\remarkname}{Remark}
\providecommand{\corollaryname}{Corollary}
\providecommand{\theoremname}{Theorem}

\makeatother

\providecommand{\problemname}{Problem}

\begin{document}
\title{Midterm Exam Microeconomics: 2021}
\author{Victor H. Aguiar }
\date{This version: 2021}

\maketitle

\begin{problem}
(30 points) Let $X$ be a nonempty and finite set of alternatives.
Consider the following bounded rational consumer. Given a menu $A\subseteq X$,
the consumer searches sequentially through the menu according to a
fixed search order on $X$. At each step in the search process, the
consumer compares the utility of the item that she sees to a fixed
threshold. If the utility of the item is above the threshold then
the consumer stops the search and picks the item. Formally, let $S\subseteq X\times X$
be a linear order on $X$, let $u:X\to\mathbb{R}$ be a utility function,
and $u^{*}\in\mathbb{R}$ be a threshold. The choice correspondence
of this bounded rational consumer is given by

\[
c_{S}(A)=\{b\in A|u(b)\geq u^{*};\text{there exists no\quad}aSb;u(a)\geq u^{*}\}
\]
 for all $A\subseteq X$, where the subscript $S$ denotes the dependence
of the choice correspondence of the linear order that models the search
process. 
\end{problem}
(a) Show that the choice correspondence $c_{S}$ defined above is
in fact a choice function. 

(b) Show that the choice correspondence $c_{S}$ satisfies WARP (define
WARP using the notes and textbook). 

(c) Assume you observe a dataset $(2^{X}\setminus\emptyset,c)$ (such
that $c:2^{X}\setminus\emptyset\to2^{X}\setminus\emptyset$ is an
observed choice function -possibly different from $c_{BR}$-) that
satisfies WARP. Show that there is a bounded rational consumer characterized
by a triple $(S,u,u^{*})$ a linear order on $X$, a utility function
and a thresholds as in (a) such that $c(A)=c_{BR}(A)$ for all $A\in2^{X}\setminus\emptyset$. 

d) Assume now that there is a distribution over search orders. Denote
the collection of all search orders/linear orders on $X\times X$
by $\mathcal{S}$. The distribution over search orders is $\pi\in\Delta(\mathcal{S})$.
The probability of choice of item $a$in menu $A$ such is given by
$\rho_{A}(a)$ where $a\in A$ and $\rho_{A}\in\Delta(A)$. Let the
stochastic version of the bounded rational model above generate $\rho_{A}$
such that:

\[
\rho_{A}(a)=\sum_{S\in\mathcal{S}}\pi(S)1(a=c_{S}(A)).
\]

Show that if you have two menus $A\subseteq B$ and $a\in A\cap B$,
it has to be that $\rho_{B}(a)\leq\rho_{B}(a)$ (i.e., regularity). 

\begin{problem}
(30 points) Let $X$ be a nonempty and finite set of alternatives.
Consider the following stochastic bounded rational consumer called
``stochastic limited consideration (SLC)''. 

Assume there is a default alternative $o\notin X$, that is always
present in any given menu. Let $X^{*}=X\cup\{o\}$. Also let the collection
of menus that contain $o$ be defined as $\mathcal{A}$, such that
$A\in\mathcal{A}$ if $A\subseteq X^{*}$ and $o\in A$. Given a menu
$A\in\mathcal{A}$, the probability of this consumer of choosing an
alternative $a$ in menu $A$, denoted by $\rho_{A}(a)$, where $\rho_{A}\in\Delta(A)$,
is given by the following decision algorithm: First the consumer is
endowed with a strict preference relation $\succ$ defined over $X^{*}$
with the restriction that for any $x\in X$, it must be that $x\succ o$
(i.e., the default is the worst item). The consumer is also endowed
with a distribution over mental categories. Formally, mental categories,
$\mathcal{D},$ is the collection all subsets of $X^{*}$. At each
decision trial, the consumer draws a mental category from $\mathcal{D}$
with probability $m\in\Delta(\mathcal{D})$ with the restriction that
$m(D)=0$ if $o\notin D$ (i.e., the probability of drawing a category
that does not have the default is zero). Then the consumer forms a
``consideration set'' by taking the intersection of menu $A$ and
$D$ such that the consideration set is equal to $D\cap A$. Then
the consumer maximizes her preferences $\succ$ on the consideration
set $D\cap A$. Finally, she chooses the item that maximizes her preferences
on $D\cap A$. The resulting probability of this decision algorithm
is:

\[
\rho_{A}(a)=\sum_{D\in\mathcal{D}}m(D)1(a\succ b\forall b\in(D\cap A),b\neq a),
\]

where $1(\cdot)=1$ when the argument is true and zero otherwise. 
\end{problem}
\begin{itemize}
\item A complete stochastic dataset, $\rho$, is a collection of $\rho_{A}$
for all menus $A\in\mathcal{A}$.
\item We say that $a$ is stochastically revealed preferred to $b$ (i.e.,
$aRb$) when $p(b,A\cup\{a\})<p(b,A)$. 
\item We say $R$ is a acyclic if there is no integer $n\geq2$ and $a_{1},\cdots,a_{n}\in X$
such that $a_{i}\succ a_{i+1}$ for $i=1,\cdots,n-1$ and $a_{n}\succ a_{1}$. 
\item We say that a complete stochastic dataset, $\rho$, is regular if
for any pair of menus $A,B\in\mathcal{A}$ such that $A\subseteq B$
it must be that $\rho_{B}(a)\leq p_{A}(a)$ for any $a\in A\cap B$. 
\end{itemize}
a) Show that a complete stochastic dataset $\rho$ that is generated
by a SLC, as described above, implies that the stochastic revealed
preference relation $R$ is acyclic. 

b) Show that a complete stochastic dataset $\rho$ that is generated
by a SLC, implies that $\rho$ is regular. 

c) Let $\rho$ be generated by a SLC with the additional restriction
that $m(X^{*})=1$ and zero otherwise. This means that this consumer
always considers all alternatives. 

Show that the choice correspondence defined by: 

\[
c(A)=\{a\in A:\rho_{A}(a)=1\},
\]

is a choice function (i.e., single-valued), and that $c(A)$ satisfies
SARP (first define SARP as seen in class and in the problem sets). 

d) Show that a complete stochastic dataset $\rho$ that is generated
by a SLC, satisfies the ASRP (Axiom of Stochastic Revealed Preference)
as defined in the notes. (Hint: use the equivalence theorem between
the ASRP and random utility and define ASRP using the notes.).

\begin{problem}
(Modeling question). (30 points) Modeling the ``Endowment Effect''.
I am going to present to you a summary of an experiment showing a
behavioral bias called the Endowment Effect. The task is to write
a decision model algorithm to explain the Endowment Effect. 
\end{problem}

\subsubsection*{The Endowment Effect Experiment. }

Kahneman, Knetch and Thaler {[}1990{]}, did an experiment with the
following features:
\begin{itemize}
\item 44 subjects 
\item 22 subjects given mugs 
\item The other 22 subjects given nothing 
\item Subjects who owned mugs asked to announce the price at which they
would be prepared to sell mug 
\item Subjects who did not own mug announced price at which they are prepared
to buy mug 
\item Experimenter figured out prices at which supply of mugs equals demand.
\end{itemize}
If our subjects are rational consumers then the prediction in this
market is:
\begin{itemize}
\item Prediction for rationality: As mugs are distributed randomly, we should
expect half the mugs (11) to get traded.
\item Explanation: Consider the group of mug lovers (i.e. those that have
valuation above the median), of which there are 22. 
\begin{itemize}
\item Half of these should have mugs, and half should not. 
\item The 11 mug haters that have mugs should trade with the 11 mug lovers
that do not. 
\end{itemize}
\end{itemize}
The \textbf{Experiment Outcome }however differs from the \textbf{Prediction},
and we call this the \textbf{Endowment Effect.}
\begin{itemize}
\item In 4 sessions, the number of trades was 4,1,2 and 2 (respectively
per session). 
\item Median seller valued mug at \$5.25 
\item Median buyer valued mug at \$2.75 
\item Willingness to pay/willingness to accept gap 
\item \textbf{Subject's preferences seem to be affected by whether or not
their reference point was owning the mug.}
\end{itemize}
\begin{enumerate}
\item Explain why the Endowment Effect is not consistent with a traditional
model of rationality. Hint: The answer has less to do with the idea
of revealed preferences, but more with the idea of stability of preferences.
Write down the model of trade formally. You can assume a parametric
distribution of preferences, make as simple as possible (uniform distribution). 
\item Propose a decision algorithm that explains the endowment effect given
the trading scheme of the experiment. 
\item Calibrate your model parameters to reproduce the experiment outcome. 
\end{enumerate}

\subsection*{Grading Criterion}
\begin{enumerate}
\item Work group is not allowed, if I find out that several of you have
essentially the same model, you will receive the total grade of this
question divided by the number of students that share the same response.
Same goes for the answers to the other two question. I expect that
each person has their own individual answers. 
\item Every question indicates how much points they are worth out of 90. 
\item The last question will be graded on the following 5 items: Logical
consistency of the decision algorithm. Clarity of the explanation
of the decision algorithm. Degree of success in explaining the endowment
effect (calibration). Clarity in the explanation of the endowment
effect using the decision algorithm. Creativity and degree to which
you have included topics that we covered in class.
\end{enumerate}

\end{document}
